% !TEX TS-program = xelatex
% !TEX encoding = UTF-8 Unicode
% !Mode:: "TeX:UTF-8"

\documentclass{resume}
\usepackage{zh_CN-Adobefonts_external} % Simplified Chinese Support using external fonts (./fonts/zh_CN-Adobe/)
% \usepackage{NotoSansSC_external}
% \usepackage{NotoSerifCJKsc_external}
% \usepackage{zh_CN-Adobefonts_internal} % Simplified Chinese Support using system fonts
\usepackage{linespacing_fix} % disable extra space before next section
\usepackage{cite}

\begin{document}
\pagenumbering{gobble} % suppress displaying page number

\name{何苏}

\basicInfo{
  \email{hesu@pku.edu.cn} \textperiodcentered\ 
  \phone{(+86) 13031037678} \textperiodcentered }
 
%\vspace{-1ex}

\section{\faGraduationCap\  教育背景}
\datedsubsection{\textbf{北京大学} \qquad \quad \textbf{硕士} \qquad \textbf{软件与微电子学院} }{2016.09 -- 2019.06}
\datedsubsection{\textbf{北京邮电大学} \quad \textbf{本科} \qquad \textbf{计算机学院} }{2012.09 -- 2016.06}

%\datedsubsection{\textbf{中国科学院计算技术研究所(保送)}~~ \ 硕士, 计算机软件与理论, 排名: 5\%}{2014 -- 2017}

\vspace{1ex}

\section{\faUsers\ 工作/实习经历}
\datedsubsection{\textbf{猿辅导} \qquad \textbf{AI Lab} \qquad 深度学习研发工程师 }{2019.07 -- 至今}

\textbf{语法纠错任务}
\begin{itemize}
  \item 使用多步finetune,序列标注的方法来解决语法纠错任务;
  \item 使用分类模型识别错误片段,再用生成模型进行纠错等;
  \item 采用词频、PMI特征建图,在生成模型的encoder端引入GCN等;
  \item 在公开数据集Conll-2014上达到SOTA的效果.
\end{itemize}

\vspace{0.1cm}

\textbf{中文错别字识别项目}
\begin{itemize}
  \item 采用bert+dense的结构进行序列标注;
  \item 使用腐化操作,随机插入、删除、替换等产生更多数据;
  \item 最终准确率达到97\%,召回率达到72\%;
\end{itemize}
\vspace{0.1cm}

\textbf{文本风格转换任务}
\begin{itemize}
  \item 采用反向翻译,回译,改写target端语句等方式进行数据增强实验;
  \item 搭建双encoder,单deocder的模型,并用GPT2进行参数初始化;
  \item 引入CopyNet机制,以及加入Adapter层,用语言模型进行rerank等;
  \item 在公开数据集GYAFC上达到SOTA的效果,并产出论文一篇;
\end{itemize}

\datedsubsection{\textbf{科大讯飞} \qquad \textbf{哈工大讯飞联合实验室} \quad nlp算法实习生 }{2017.09 -- 2018.10}
\vspace{0.1cm}

\textbf{选择题阅读理解项目}
\begin{itemize}
  \item 将问题和候选项拼接,去外部知识库中检索相关语句;
  \item 搭建模型计算候选项之间,问题和外部知识语句之间的attention等;
  \item 在ARC数据集上达到SOTA效果;
\end{itemize}
\vspace{0.1cm}

\textbf{智能客服项目}
\begin{itemize}
  \item 拒答模块:搭建二分类模型,判别用户提出的问题是否给与回答,最终准确率为98\%;
  \item 搜索浏览模块:对于用户提出的问题,从511个文档中找到相应的文档,最终准确率为86\%;
  \item 内容详情模块:从文档中,检索出连续的句子回答问题,最终准确率为91\%;
\end{itemize}
\vspace{0.1cm}

\textbf{多轮对话项目}
\begin{itemize}
  \item 将内容详情模型迁移到ubuntu/doban对话数据集上;
  \item 采用SGD微调,多模型融合等,在douban数据集上达到SOTA效果;
\end{itemize}




\section{\faCogs\ 自我评价}
% increase linespacing [parsep=0.5ex]
对于nlp任务中的语法纠错、文本风格转换、阅读理解等任务有一定的了解,有一定的论文阅读量。能针对特定需求场景快速搭建模型,并迭代优化。
具有良好的团队沟通能力,有良好的抗压能力。
%% Reference
%\newpage
%\bibliographystyle{IEEETran}
%\bibliography{mycite}
\end{document}
